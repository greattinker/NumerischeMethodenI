\section{Formulierung der Randwertaufgabe}
Beispiel Zugstab\\
(Graphik Zugstab einfügen)\\
\subsection{Differentielle Formulierung}
Auswertung von Bilanzgleichungen an einem infinitessimalen Volumenelement\\
\begin{itemize}
\item GGW: \\\(-F_L + F_L + dF_L + qdx = 0 \\\\ \frac{dF_L}{dx} = -q \\\\ F_L' = -q\)
\item kinematische Annahme\\
	\(\epsilon = \frac{du}{dx} = u'\)
\item kinetische Annahme:\\
	\(F_L = \sigma \cdot A \)
\item[$\Rightarrow$] konstitutive Beziehung (Material)\\
	\(\sigma = E \cdot \epsilon\)
\item[$\Rightarrow$] Einsetzen in die Grundgleichung\\
	\(F_L = \sigma \cdot A = E \cdot \epsilon \cdot A \quad \Rightarrow \quad F_L' = (E \cdot A \cdot u')' = -q\)
\end{itemize}
\paragraph{Annahme: konstitutiver Querschnitt??}
\(E\,A\,u'' + q = 0 \qquad \forall x \in [0,l]\)

\begin{itemize}
	\item RB: 
		\begin{itemize}
		\item wesentliche RB: \(u(x=0) = \tilde{u} = 0 \)
		\item natürliche RB: \(F_L (x=l) = F = 0 \\ E\,A\,u' (x=l) = F = 0\)
		\end{itemize}
	\item Zuordnung \\
			\(D = E\,A (...)'' \\
			D_1 = 1 (...) \\
			D_2 = E\,A(...)' \\
			\Omega : x \in [0, l] \\
			\Gamma_1 : x = 0 \\
			\Gamma_2 : x = l \)
\end{itemize}

\subsection{Variationsformulierung}
sdsdsd

\subsection{Prinzip der virtuellen Arbeit}
\begin{itemize}
	\item elektrisches Gesamtpotential: Einschränkung auf elastische Potentiallasten
	\item[$\Rightarrow$] Prinzip der virtuellen Arbeit ist allgemein anwendbar = Prinzip der virtuellen Verschiebung
	
	\begin{itemize}
		\item virtuelle Arbeit {\boldmath\(\partial W\)} wird an einem System verrichtet (durch eine geringfügige Störung \(\partial u\) des Verschiebungsfeldes \(u\))
		\item Eigenschaften von \textbf{\boldmath\(\partial u\) = virtuelle Verschiebung}
			\begin{itemize}
				\item beliebig
				\item infinitessimal
				\item kinematisch zulässig (\(\partial u = 0\))
			\end{itemize}
		\item Gleichgewicht (GGW) entspricht der Forderung \\
			\(\partial W = 0\)\\
			\(\partial W = \partial W_{innen} + \partial W_{außen} = 0\)
	\end{itemize}
\end{itemize}
\paragraph{Beispiel: Stab}
	\(\partial W = \int G \partial \rho\,dV - \int\limits_0^l q\,du\,dx - Fdu(l)\)\\
	beliebiges Materialverhalten, keine Einschränkungen\\
	\(\Rightarrow\) GGW: \(\partial W = 0\)\\
	\(\Gamma = E \epsilon\), \(\epsilon = u'\), \(dV = Ad\)\\\\
	{\boldmath\(\partial W = \int\limits_0^l EAu' \partial u'\, dx - \int\limits_0^l q \partial u\, dx - Fdu(l) = 0\)}\\\\
	$\Rightarrow$ partielle Integration, \(\partial u(0) = 0\), sonst \(\partial u\) beliebig\\
	liefert 
	\begin{itemize}
		\item \(E A u'' + q = 0 \qquad\) DGL
		\item \(E A u'(l) -F = 0 \qquad\) natürliche Randbedingung
	\end{itemize}
